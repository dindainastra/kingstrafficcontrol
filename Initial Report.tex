\documentclass{article} \usepackage{graphicx}
\usepackage{amsmath}
\usepackage{color,soul}
\usepackage{lastpage}


\begin{document}
	
	\title{Initial Report
    \\ Group Project 7CCSMGPR} 	
    \author{King's Traffic Control}
	\maketitle
	
	 \section{Project Description} 		
        To design an efficient Traffic Simulator System to see the traffic flow and interaction under different traffic policies.
        
         \subsection{Project Aims and Strategy}
          Strategy: To complete a fully functioning Traffic Simulation System (Mandatory Level
           1 specification). Complete Level 2 and 3 of the Aims as time permits. The whole project will be designed using the agile model (scrum).
         \\The simulation will be programmed in Java with the use of GitHub repository. A 'working'
          branch will be implemented . A 'master' branch is made to act as the container of the main
           code. Each week end this 'master' branch will be updated with the latest version of the 
           'working' branch. This is done to prevent changes being made to the main branch 
           with conflicts.
			\subsubsection{Project Strategy - Scrum}
			hello\\
			\subsubsection{Project Tasks:} 					
			See Project Work breakdown structure under 1.3 Project Schedule, for when each tasks is planned to be accomplished.
            \begin{description} 				
                \item[Level 1] Mandatory Specification
                        (1) Individual vehicles
                                (a) Cars, lorries, bike, buses, coaches, motorbikes 					
                                (b) Reckless driver, cautious, normal 								
                        (2) Road network
                        (3) Entrance to road networks and exists
                        (4) Individual behaviours
                        (5) Timing of journeys - rush hours
                        (6) Emergency services - police, ambulance, fire truck
                        (7) Management policies - divert traffic or roadworks
                        (8) Time granularity 							
                  \item[Level 2] Optional Specification
                            (1) People crossings
                            (2) Politeness level
                            (3) Traffic lights
                            (4) Map scale user interface
                            (5) Congestion rate user interface
                            (6) Pause, stop, resume, play
                            (7) Speed limit option
                            (8) Scenery
                  \item[Level 3] Optional Specification
                        (1) Configurable map
                        (2) Road types
                            (a) Bridge
                            (b) Crossings
                            (c) Traffic lights mechanism
                            (d) Junctions
                            (e) Single carriageways
                            (f) Dual
                            (g) Roundabouts
                            (h) Tolls gates
                            (i) Uphill/downhill
                            (j) Ramps
                        (3) Parking (street)
                        (4) Changing lanes
                        (5) Weather conditions
                \end{description}


                \subsection{Project Design}
                \textbf{Class Diagram:}\\
                \includegraphics[width=150mm]{ClassDiagram.png}
				  
                \subsection{Project Schedule}
                \subsubsection{Rudimentary Timetable}
                    The figure bellow shows the Work Breakdown Structure for our project. This is represented to show how the scrum agile method has been incorporated into the project.\\
                    \\The figure bellow is the GANTT Chart. This shows the critical tasks the group must accomplish for it to remain on task. From this, a risk management section is also created as shown below on 2.3 Rish Management. 		
                \subsection{Initial Progression}
                \begin{description}
				 \item[Level 1] GUI Based:
                    \begin{enumerate}
                        \item GUI Road network, buttons, slider and so on 
                        \item Object classes
                        \item Using swing Library
                    \end{enumerate}
                 \item[Level 2] Algorithm:
                    \begin{enumerate}
                      \item Network node libraries
                      \item Node diagram
                      \item Pseudo codes, Implementation of algorithms.
                    \end{enumerate}
                 \item[Level 3] Management:
                    \begin{enumerate}
                      \item Everything is represented as a vehicle.
                      \item Vehicle behaviour depends on the priority.
                      \item Approach based on attributes of an object in different threads.
                    \end{enumerate}
                \end{description}


	\section{Project Organisation}
        \subsection{Project Role and Responsibilities}
            As shown in Figure 1, on the 1.3.1 Rudimentary Timetable section of this 
            report, level 1 of the project will be the first to be implemented. Level 
            1 of the project has been split into three parts. As a group of six, two
             persons are allocated for each part. The task distribution are as follows: 			
            \begin{itemize}
                \item GUI, Objects and Report: Sai Kurdukar and Daniella Bacud
                \item Algorithms: Patrick Martins - Yedenu, Gregory Pavlidis and (Daniella Bacud)
                \item Management Policies: Dyah Damapuspita and Constantinos Hasikos
            \end{itemize}
        \subsection{Communication}
            Physical meetings are held on Thursday at 13:00 or Fridays at 13:30 each week. 
            Additional meetings may be conducted as required. Main social network contact
             is done through Whatsapp, email, Trello, Slack and GitHub.
            \\Conflicts and disagreements are solved through the majority vote of the group.
            \\The tasks are segregated in such a way that each member of the team has equal
             workload and contribution to the project. If in cases where tasks cannot be 
             fairly distributed, members with less load will take up the workload required
              to complete the reports and/or presentations. 		
              Documentation are recorded on the Google drive. 		
        \subsection{Risk Management} 		
            At each weekly meeting, weekly tasks is set and the progress of the team 
            is marked. In situations whereby a member is having difficulties in completing 
            their tasks, it will be immediately identified and resolved either through 
            mitigation: the help of another member or through reallocation of tasks, 
            or through avoidance. 	
        \section{Reference}
            IEEE standard
						
\end{document} 
